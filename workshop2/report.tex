\documentclass[10pt]{article}
 \usepackage{graphicx}
 \usepackage[tight]{subfigure}
 \usepackage{amssymb}
 \usepackage{mathtools}
 \usepackage{amsmath}
 \usepackage{clrscode}
 \usepackage{algorithmicx}
\usepackage{algorithm}
\usepackage[noend]{algpseudocode}
\usepackage{xcolor}
\usepackage{listings}
\usepackage{hyperref}
\usepackage{color}

\definecolor{dkgreen}{rgb}{0,0.6,0}
\definecolor{gray}{rgb}{0.5,0.5,0.5}
\definecolor{mauve}{rgb}{0.58,0,0.82}

\lstset{frame=tb,
  language=Python,
  aboveskip=3mm,
  belowskip=3mm,
  showstringspaces=false,
  columns=flexible,
  basicstyle={\small\ttfamily},
  numbers=none,
  numberstyle=\tiny\color{gray},
  keywordstyle=\color{blue},
  commentstyle=\color{dkgreen},
  stringstyle=\color{mauve},
  breaklines=true,
  breakatwhitespace=true,
  tabsize=3
}
\hypersetup{
    colorlinks=true,
    linkcolor=blue,
    filecolor=magenta,      
    urlcolor=cyan,
    pdftitle={Overleaf Example},
    pdfpagemode=FullScreen,
    }

\urlstyle{same}


\algrenewcommand{\algorithmiccomment}[1]{ \{ #1 \} }

\newif\ifboldnumber
\newcommand{\boldnext}{\global\boldnumbertrue}

 \setlength{\oddsidemargin} {0 in}
\setlength{\textwidth} {6.5 in}
\setlength{\textheight} {8.5 in}
\setlength{\topmargin} {-0.5 in}
\def\h{\hspace{-.9pt}{\_}}
\newcommand{\reminder}[1]{ [[[ \marginpar{\mbox{$<==$}} #1 ]]] }
\newcommand{\eatreminders}[0]{\renewcommand{\reminder}[1]{}}
\newcommand\quoted[1]{``#1''}
\author{Brian Zhu\\
\url{https://github.com/zhubiii/CSE360} }
\date{}
\title{CSE 360: Workshop 2\\
}
\begin{document}

\maketitle
 
\begin{enumerate}
   \item
   \begin{itemize}
     \item The camera takes a 14x8 picture. $(u_0=7,v_0=4)$ represent the optical center.
     \item We simply use the equations that convert plane coordinates to pixel coordinates $u=u_0+\frac{k_ufX_c}{Z_c}$ and $v=v_0+\frac{k_vfY_c}{Z_c}$ and $(u=12,v=6)$
     \begin{itemize}
        \item $12=7+\frac{(21)(1)X_c}{15} \implies X_c=\frac{75}{21}=3.57$ 
        \item $6=4+\frac{(11)(1)Y_c}{15} \implies Y_c=\frac{30}{11}=2.73$
     \end{itemize}
     \item We know that a rotation matrix can transform a point from one frame to another
     \item $^qR_cp_c=p_q \implies
      \begin{bmatrix}
       cos(\pi/2) & 0 & sin(\pi/2) \\
       0 & 1 & 0 \\
       -sin(\pi/2) & 0 & cos(\pi/2)
      \end{bmatrix}
      \begin{bmatrix}
       cos(-\pi/2) & -sin(-\pi/2) & 0 \\
       sin(-\pi/2) & cos(-\pi/2) & 0 \\
       0 & 0 & 1
      \end{bmatrix}
      \begin{bmatrix}
        3.57\\
        2.73\\
        15
      \end{bmatrix}
      =
      \begin{bmatrix}
        15\\
        -3.57\\
        -2.73
      \end{bmatrix}
      $
   \end{itemize}
  \item We want to find $d_g$ and we know that the inverse of a Rotation matrix gives us the rotation matrix with the frames flipped with respect to each other
  \begin{itemize}
    \item $^qR^{T}_g {=} {^gR_q} \implies ^gR_qd_q=d_g$
    \item $
      \begin{bmatrix}
       cos(-\pi/6) & -sin(-\pi/6) & 0 \\
       sin(-\pi/6) & cos(-\pi/6) & 0 \\
       0 & 0 & 1
      \end{bmatrix}^T
      \begin{bmatrix}
        15\\
        -1\\
        -3
      \end{bmatrix}
      =
      \begin{bmatrix}
        \frac{15\sqrt{3} +1}{2}\\
        \frac{-\sqrt{3} +15}{2}\\
        -3
      \end{bmatrix}
      $
  \end{itemize}
  \item H-Transformation matrices have the same frame flipping inverse properties as rotation matrices  $^wT^T_q=^qT_w \implies {^wT^T_q}^wT_g=^qT_g$
  \item
  \begin{itemize}
    \item Car 1
    \begin{itemize}
      \item For car 1, the rotation matrix and the position are both dependent on time
      \item The rotation matrix, it will always rotate about the z-axis. The
      position is a 180-degree rotation CCW which is $\pi/2$ according to right-hand rule
      Therefore, the rotation at time $t=0$ is Rot($z, \pi/2$)
      \item we know also need to get the rotation as a function of time. $v_1*t$ describes
      it movement along the circumference. This can be referred to as the arc length. From This
      we know how much theta has changed due to the arc length equation $\frac{ds}{r}=d\theta$ 
      \item Therefore, the rotation matrix w.r.t. the \{0\} frame is Rot($z, \pi/2-\frac{(v_1*t)}{2}$)
      \item To get the change in position we need to first define the position relative to the \{0\} frame at time 
      $t=0$. This is simply $(1+2cos(\pi/2),1+2sin(\pi/2),3)$ 
      \item To find the position as a function of time we utilize the same arc length formula
      as we did for the rotation matrix which gives us the (x,y) position of a circle radius 2 
      centered at (0,0). To offset we add 1 to x and y. Therefore, the position over time is $(1+2cos(\pi/2-\frac{v_1*t}{2}),1+2sin(\pi/2-\frac{v_1*t}{2}),3)$
      \item Finally, to obtain the transformation matrix we combine our rotation and postion
      \item $^0T_1=\begin{bmatrix}
        cos(\pi/2-\frac{v_1*t}{2}) & -sin(\pi/2-\frac{v_1*t}{2}) & 0 & 1+2cos(\pi/2-\frac{v_1*t}{2})\\
        sin(\pi/2-\frac{v_1*t}{2}) & cos(\pi/2-\frac{v_1*t}{2}) & 0 & 1+2sin(\pi/2-\frac{v_1*t}{2}) \\
        0 & 0 & 1 & 3 \\
        0 & 0 & 0 & 1
      \end{bmatrix}
      $
    \end{itemize}
    \item Car 2
    \begin{itemize}
      \item For car 2, the rotation is constant about the z-axis clockwise which is negative according to right-hand rule, so the rotation matrix
      is described as Rot($z, -\pi/4$)
      \item The starting position of car 2 at time $t=0$ is 
      $(2cos(3\pi/4),2sin(3\pi/4),0)$ w.r.t. the center of the table. Multiplied by two due to the radius and 0 because its level with the tabletop
      \item To get it with respect to frame \{0\} we simply subtract 1 from x and y
      such that the starting position of car 2 is $(2cos(3\pi/4)-1, 2sin(3\pi/4)-1, 3)$. Note that the z is 3 now because that is the height of the table
      \item the radius changes, however, so we need to put the radius term in terms of velocity of time. $((2-v_2*t)cos(3\pi/4)-1, (2-v_2*t)sin(3\pi/4)-1, 3)$
      \item Finally, to obtain the transform matrix we combine our rotation matrix and position vector 
      \item $^0T_2=\begin{bmatrix}
        cos(-\pi/4) & -sin(-\pi/4) & 0 & (2-v_2*t)cos(3\pi/4)-1\\
        sin(-\pi/4) & cos(-\pi/4) & 0 & (2-v_2*t)sin(3\pi/4)-1 \\
        0 & 0 & 1 & 3 \\
        0 & 0 & 0 & 1
      \end{bmatrix}
      $
    \end{itemize}
    \item to find the homogeneous matrix $^1T_2$ as a function of time we take
    the inverse of $^0T_1$ and multiply it by $^0T_2$ such that $^1T_0{^0T_2}=^1T_2$
    The inverse of a transform matrix is the transpose of the rotation matrix and then
    multiplying the position vector by the negative transpose of the rotation matrix $-R^Tp$
    \item $-R^Tp=\begin{bmatrix}
      cos(\frac{v_1t}{2}) + sin(\frac{v_1t}{2}) + 2 \\
      -cos(\frac{v_1t}{2}) + sin(\frac{v_1t}{2})  \\
      3
    \end{bmatrix}$
    \item $^1T_2=\begin{bmatrix}

    \end{bmatrix}
    $ Its just the matrix product as described above.
  \end{itemize}
\end{enumerate}
\end{document}